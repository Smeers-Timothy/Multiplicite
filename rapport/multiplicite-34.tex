\documentclass[a4paper, 11pt, oneside]{article}

\usepackage[utf8]{inputenc}
\usepackage[T1]{fontenc}
\usepackage[french]{babel}
\usepackage{array}
\usepackage{shortvrb}
\usepackage{listings}
\usepackage[fleqn]{amsmath}
\usepackage{amsfonts}
\usepackage{fullpage}
\usepackage{enumerate}
\usepackage{graphicx}             % import, scale, and rotate graphics
\usepackage{subfigure}            % group figures
\usepackage{alltt}
\usepackage[hidelinks]{hyperref}  
\hypersetup{
	colorlinks=true,
	linkcolor=blue,
	filecolor=magenta,      
	urlcolor=cyan,
	pdfpagemode=FullScreen,
}
\usepackage{url}
\usepackage{indentfirst}
\usepackage{eurosym}
\usepackage{listings}
\usepackage{color}
\usepackage[table,xcdraw,dvipsnames]{xcolor}

% Change le nom par défaut des listing
\renewcommand{\lstlistingname}{Codingstyle}

% Change la police des titres pour convenir à votre seul lecteur
\usepackage{sectsty}
\allsectionsfont{\sffamily\mdseries\upshape}
% Idem pour la table des matière.
\usepackage[nottoc,notlof,notlot]{tocbibind}
\usepackage[titles,subfigure]{tocloft}
\renewcommand{\cftsecfont}{\rmfamily\mdseries\upshape}
\renewcommand{\cftsecpagefont}{\rmfamily\mdseries\upshape}

\definecolor{mygray}{rgb}{0.5,0.5,0.5}
\newcommand{\coms}[1]{\textcolor{MidnightBlue}{#1}}

\lstset{
language=C, % Utilisation du langage C
commentstyle={\color{MidnightBlue}}, % Couleur des commentaires
frame=single, % Entoure le code d'un joli cadre
rulecolor=\color{black}, % Couleur de la ligne qui forme le cadre
stringstyle=\color{RawSienna}, % Couleur des chaines de caractères
numbers=left, % Ajoute une numérotation des lignes à gauche
numbersep=5pt, % Distance entre les numérots de lignes et le code
numberstyle=\tiny\color{mygray}, % Couleur des numéros de lignes
basicstyle=\tt\footnotesize,
tabsize=3, % Largeur des tabulations par défaut
keywordstyle=\tt\bf\footnotesize\color{Sepia}, % Style des mots-clés
extendedchars=true,
captionpos=b, % sets the caption-position to bottom
texcl=true, % Commentaires sur une ligne interprétés en Latex
showstringspaces=false, % Ne montre pas les espace dans les chaines de caractères
escapeinside={(>}{<)}, % Permet de mettre du latex entre des <( et )>.
inputencoding=utf8,
literate=
{á}{{\'a}}1 {é}{{\'e}}1 {í}{{\'i}}1 {ó}{{\'o}}1 {ú}{{\'u}}1
{Á}{{\'A}}1 {É}{{\'E}}1 {Í}{{\'I}}1 {Ó}{{\'O}}1 {Ú}{{\'U}}1
{à}{{\`a}}1 {è}{{\`e}}1 {ì}{{\`i}}1 {ò}{{\`o}}1 {ù}{{\`u}}1
{À}{{\`A}}1 {È}{{\`E}}1 {Ì}{{\`I}}1 {Ò}{{\`O}}1 {Ù}{{\`U}}1
{ä}{{\"a}}1 {ë}{{\"e}}1 {ï}{{\"i}}1 {ö}{{\"o}}1 {ü}{{\"u}}1
{Ä}{{\"A}}1 {Ë}{{\"E}}1 {Ï}{{\"I}}1 {Ö}{{\"O}}1 {Ü}{{\"U}}1
{â}{{\^a}}1 {ê}{{\^e}}1 {î}{{\^i}}1 {ô}{{\^o}}1 {û}{{\^u}}1
{Â}{{\^A}}1 {Ê}{{\^E}}1 {Î}{{\^I}}1 {Ô}{{\^O}}1 {Û}{{\^U}}1
{œ}{{\oe}}1 {Œ}{{\OE}}1 {æ}{{\ae}}1 {Æ}{{\AE}}1 {ß}{{\ss}}1
{ű}{{\H{u}}}1 {Ű}{{\H{U}}}1 {ő}{{\H{o}}}1 {Ő}{{\H{O}}}1
{ç}{{\c c}}1 {Ç}{{\c C}}1 {ø}{{\o}}1 {å}{{\r a}}1 {Å}{{\r A}}1
{€}{{\euro}}1 {£}{{\pounds}}1 {«}{{\guillemotleft}}1
{»}{{\guillemotright}}1 {ñ}{{\~n}}1 {Ñ}{{\~N}}1 {¿}{{?`}}1
}
\newcommand{\tablemat}{~}

%%%%%%%%%%%%%%%%% TITRE %%%%%%%%%%%%%%%%
% Complétez et décommentez les définitions de macros suivantes :
\newcommand{\intitule}{Construction de Programme}
\newcommand{\GrNbr}{34}
\newcommand{\PrenomUN}{Timothy}
\newcommand{\NomUN}{Smeers}
\newcommand{\PrenomDEUX}{Soline}
\newcommand{\NomDEUX}{Lèbre}
% Décommentez ceci si vous voulez une table des matières :
\renewcommand{\tablemat}{\tableofcontents}

%%%%%%%% ZONE PROTÉGÉE : MODIFIEZ UNE DES DIX PROCHAINES %%%%%%%%
%%%%%%%%            LIGNES POUR PERDRE 2 PTS.            %%%%%%%%
\title{INFO0947: \intitule}
\author{Groupe \GrNbr : \PrenomUN~\textsc{\NomUN}, \PrenomDEUX~\textsc{\NomDEUX}}
\date{\today}
\begin{document}
\maketitle
\newpage
\tablemat
\newpage
%%%%%%%%%%%%%%%%%%%% FIN DE LA ZONE PROTÉGÉE %%%%%%%%%%%%%%%%%%%%

% Section Sous-problème
\section{Sous-problèmes}
\begin{itemize}
	\item[] $SP_1 :$ Balayer le tableau
	\begin{itemize}
		\item[\small $\bullet$] Input : Un tableau d'entier de taille N
		\item[\small $\bullet$] Output : $\backslash$
		\item[\small $\bullet$] Objets utilisé(s) : 
		\begin{itemize}
			\item[$\star$] unsigned int i;
			\item[$\star$] unsigned int j;
			\item[$\star$] int $T\lbrack N \rbrack;$
		\end{itemize}
	\end{itemize}
	\medskip
	\item[] $SP_2$ : Trouver le plus grand nombre présent dans le tableau et incrémenté une variable 
				   de 1 pour chaque occurence trouvée
	\begin{itemize}
		\item[\small $\bullet$] Input : Un tableau d'entier de taille N
		\item[\small $\bullet$] Output : Le maximum et le nombre d'occurence(s)
		\item[\small $\bullet$] Objets utilisé(s) : 
		\begin{itemize}
			\item[$\star$] int max;
			\item[$\star$] unsigned int occurence;
			\item[$\star$] unsigned int $T\lbrack N \rbrack;$ 
		\end{itemize}
	\end{itemize}
	\medskip
	\item[] \underline{Lien entre les $SP_s$} $: SP_1 \subset SP_2$
\end{itemize}

% Section Notations
\section{Notations}
\begin{itemize}
	\item[] $\#i \bullet (i \in \lbrack 0, N-1 \rbrack) \hspace{1mm} \mid \hspace{1mm} (max_i \hspace{1mm} T\lbrack N \rbrack)$
\end{itemize}

\section{Précondition \& postcondition}
\begin{itemize}
	\item[] Préconditon : $T\lbrack N \rbrack \neq NULL \hspace{2mm} \&\& \hspace{2mm} N > 0 \rightarrow assert(T \hspace{1mm} != NULL \hspace{2mm} \&\& \hspace{2mm} N > 0);$
	\medskip
	\item[] Postcondition : 
	\begin{itemize}
		\item[\small $\bullet$] $T = T_0 \hspace{2mm} \&\& \hspace{2mm} N = N_0 \hspace{2mm} \&\& \hspace{2mm} max \hspace{1mm}!= 0 \hspace{2mm} 
		\&\& \hspace{2mm}  occurence > occurence_0$
		\item[\small $\bullet$] $max$ contient la valeur du plus grand nombre présent dans $T\lbrack N \rbrack$ 
		\item[\small $\bullet$] $occurence$ contient le nombre d'occurence de max présent dans $T\lbrack N \rbrack$  
	\end{itemize}				
\end{itemize}

% Section Invariant graphique
\newpage
\section{Invariant graphique}

\begin{figure}[!h]
\centering
\begin{tabular}{lllll}
\multicolumn{1}{l|}{}      & \multicolumn{1}{l|}{0}                                                                      & \multicolumn{1}{l|}{i}                                       & \multicolumn{1}{l|}{j+1}                                                                       & N \\
\multicolumn{1}{l|}{T :}   & \multicolumn{1}{l|}{\cellcolor[HTML]{34CDF9}Déja balayer}                                   & \multicolumn{1}{l|}{\cellcolor[HTML]{FE0000}Reste à balayer} & \multicolumn{1}{l|}{\cellcolor[HTML]{34CDF9}Déja balayer}                                    &   \\ \cline{2-4}
                           & - max contient la valeur maximum                                                            &                                                              & - max contient la valeur de maximum                                                          &   \\
                           \medskip
                           & \begin{tabular}[c]{@{}l@{}}- occurence contient le nombre\\ d'occurence de max\end{tabular} &                                                              & \begin{tabular}[c]{@{}l@{}}- occurence contient le nombre \\ d'occurence de max\end{tabular} &  
\end{tabular}
\caption{Invariant graphique}
\label{fig:exemple}
\end{figure}

\begin{flushleft}
	\begin{enumerate}
		\item ${ G.B = i <= j }$
		\item ${ C.A = i > j }$ 
		\item ${ F_t = j-i }$
	\end{enumerate}
\end{flushleft}

% Sub-section complexité
\subsection{Règle de complexité}
\begin{itemize}
	\item[$R_1 :$] Instruction de base \newline
					$T(.) = 1$ \newline
	\item[$R_2 :$] Séquence d'instruction \newline
					$T(.) = T_1(.) + T_2(.) + \ldots + T_k(.)$ \newline
	\item[$R_3 :$] if \newline
					$T(.) = max(T_1(.), T_2(.))$ \newline
	\item[$R_4 :$] Switch \newline
					$T(.) = max(T1(.), T2(.), \ldots, T_k(.), T_{k+1}(.))$\newline
	\item[$R_5 :$] Boucle \newline
					$T(.) = T_1(.) + T_2(.) + \ldots + T_k(.)$ \newline 
					$ = \sum_{i=1}^{k} \underbrace{T_i(.)}_{\text{gardien de boucle}}$ \newline
	\item[$R_6 :$] Programme \newline

\end{itemize}

\newpage
% Sub-section Calcul de la complexité du code
\subsection{Calcul de la complexité du code}
	\begin{flushleft}
		\text{int multiplicite ( int *T , const int N , int *max ) \{ }\\
		\text{assert(T != NULL \&\& N > 0);} 
	\newline
		
%--------------T(A)-----------------%
	$\left.
    \begin{array}{}
		\text{unsigned int i = 0;}\\
		\text{unsigned int j = N-1;}\\
		\text{unsigned int occurence = 0;}\\
		\text{*max = T[0];}
    \end{array}
	\right\} T(A) $\\ \\
%--------------END T(A)-----------------%

%--------------T(B')-----------------%
	$\left.
	\begin{array}{llll}
		\text{while (i <= j) \{} \\ \\
%--------------T(B_a')-----------------%
	\left.
	\begin{array}{}
%--------------T(B')-----------------%
	\left.
	\begin{array}{}
			\hspace{4mm} \text{if (T[i] > *max || T[j] > *max)\{} \\ 
%--------------T(B_a1'')-----------------%
	\left.
	\begin{array}{}
				\hspace{8mm} \text{if( T[i] == T[j])\{} \\
					\hspace{12mm} \text{*max = T[i];} \\
					\hspace{12mm} \text{occurence = 2;} 
	\end{array}
	\right\} T(B_{a1}'') \\ \\
%--------------END T(B_a1'')-----------------%
	
%--------------T(B_a2'')-----------------%
	\left.
	\begin{array}{}
				\hspace{8mm} \text{\}else if(T[i] > T[j])\{} \\
					\hspace{12mm} \text{*max = T[i];} \\
					\hspace{12mm} \text{occurence = 1;} 
	\end{array}
	\right\} T(B_{a2}'') \\ \\
%--------------END T(B_b2'')-----------------%

%--------------T(B_a3'')-----------------%
	\left.
	\begin{array}{}
				\hspace{8mm} \text{\}else\{} \\
					\hspace{12mm} \text{*max = T[j];} \\
					\hspace{12mm} \text{occurence = 1;} 
	\end{array}
	\right\} T(B_{a3}'') \\ 
%--------------END T(B_a3'')-----------------%
				\hspace{8mm}\text{\}} 
	\end{array}
	\right\} T(B_a') \\ \\
%--------------END T(B_a')-----------------%

%--------------T(B_b')-----------------%
	\left.
	\begin{array}{}
			\hspace{4mm} \text{\}else if(*max == T[i] || *max == T[j])\{}  \\	
%--------------T(B_b1'')-----------------%
	\left.
	\begin{array}{} 
				\hspace{8mm} \text{if(i == j)} \\
					\hspace{12mm}\text{occurence++;} 
	\end{array}
	\right\} T(B_{b1}'') \\
%--------------END T(B_b1'')-----------------%

%--------------T(B_b2'')-----------------%
	\left.
	\begin{array}{} 
				\hspace{8mm} \text{else if(T[i] == T[j])} \\
					\hspace{12mm}\text{occurence +=2;} 
	\end{array}
	\right\} T(B_{b2}'') \\
%--------------END T(B_b2'')-----------------%

%--------------T(B_b3'')-----------------%
	\left.
	\begin{array}{} 
				\hspace{8mm} \text{else} \\
					\hspace{12mm} \text{occurence++;} 
	\end{array}
	\right\} T(B_{b3}'') \\
%--------------END T(B_b3'')-----------------%
	\end{array}
	\right\} T(B_b') \\ \\
%--------------END T(B_b')-----------------%
			\hspace{4mm} \text{\}} \\
	\end{array}
	\right\} T(B') \\ \\
%--------------END T(B')-----------------%
%--------------T(B_c')-----------------%
	\left.
	\begin{array}{} 
			\hspace{4mm} \text{i++;} \\
			\hspace{4mm} \text{j--;} 
	\end{array}
	\right\} T(B_c') \\
%--------------END T(B_c')-----------------%
		\text{\}} 
	\end{array}
	\right\} T(B)$
%--------------END T(B)-----------------%
		
		\text{return(occurence);} \\
		\text{\}}
	\end{flushleft}
	
\begin{itemize}
	
	\item[$\small \bullet$] $T(A) = 1$ \text{Par application de la règle 1}
	\item[$\small \bullet$] $T(B_{a1}'') = 2 $ \text{Par application de la règle 1, 2, 3} 
	\item[$\small \bullet$] $T(B_{a2}'') = 2 $ \text{Par application de la règle 1, 2, 3} 
	\item[$\small \bullet$] $T(B_{a3}'') = 2 $ \text{Par application de la règle 1, 2, 3} 
	\item[$\small \bullet$] $T(B_a') = max(T(B_{a1}''), T(B_{a2}''), T(B_{a3}''))$ \text{Par application de la règle 2 et 3}\newline
							$T(B_a') = 2$
	\item[$\small \bullet$] $T(B_{b1}'') = 1 $ \text{Par application de la règle 1 et 2 et 3} 
	\item[$\small \bullet$] $T(B_{b2}'') = 1 $ \text{Par application de la règle 1 et 2 et 3} 
	\item[$\small \bullet$] $T(B_{b3}'') = 1 $ \text{Par application de la règle 1 et 2 et 3} 
	\item[$\small \bullet$] $T(B_b') = max(T(B_{b1}''), T(B_{b2}''), T(B_{b3}''))$ \text{Par application de la règle 2 et 3}\newline
							$T(B_b') = 1$
	\item[$\small \bullet$] $T(B_c') = 2$ \text{Par application de la règle 1}
	\item[$\small \bullet$]	$T(B') = max(T(B_a'), T(B_b'))$ \text{Par application de la règle 2 et 3}\newline
							$T(B') = 2$
	\item[$\small \bullet$] $T(B) = \sum_{i=1}^{k} T_i(.)$ \text{Par application de la règle 5 } \newline
							$T(B) = n/2$
	\item[$\small \bullet$] $T(n) = T(A) + T(B)$ \text{Par application de la règle 2 et 1} 
							$T(n) = (n/2) + 2$
\end{itemize}



% Section Invariant formel
\newpage
\section{Invariant formel}
$ 0 \leq i \leq N-1 \wedge max = max_i \hspace{1mm} T\lbrack N \rbrack \wedge occurence = \#i \bullet(i\mid max) $

% Section code
\section{Code}	
	\subsection{main.c}	
		\begin{lstlisting}[caption={main.c}]
#include <stdio.h>
#include "multiplicite.h"

int main(){

	int T [9] = {13 , 16 , 19 , 17 ,19 , 12 , 2, 4, 10};
	int max ;

	int occurrences = multiplicite(T, 9, &max);
	printf("%d - %d\n", occurrences, max);
}

		\end{lstlisting}
\newpage
	\subsection{multiplicite.h}
		\begin{lstlisting}[caption={multiplicite.h}]
#ifndef MULTIPLICITE_H_
#define MULTIPLICITE_H_

/**
 * @fn int multiplicite(int*, const int, int*)
 * @brief Obtain, for T, the greatest value of T as well as the number
 * of occurrences of this value in T.
 *
 * @pre T != NULL && N > 0
 * @post T = T0 && N = N0 && max != 0 && occurence > occurence0
 * 		 max contains the value of the largest number present in T [N]
 * 		 occurrence contains the number of occurrences of max present in T [N]
 * @param T an array with N integer values
 * @param N table size T
 * @param max contains the value of the largest number present in T [N]
 * @return The number of occurrences of the maximum in T.
 */
int multiplicite ( int *T , const int N , int *max );

#endif MULTIPLICITE_H_
		\end{lstlisting}
		
\newpage
	\subsection{multiplicite.c}
		\begin{lstlisting}[caption={multiplicite.c}]
#include <stdio.h>
#include <assert.h>
#include "multiplicite.h"

int multiplicite ( int *T , const int N , int *max ) {
	assert(T != NULL && N > 0);

	unsigned int i = 0;
	unsigned int j = N-1;
	unsigned int occurence = 0;
	*max = T[0];

	while (i <= j) {

		if (T[i] > *max || T[j] > *max){
			if( T[i] == T[j]){
				*max = T[i];
				occurence = 2;

			}else if(T[i] > T[j]){
				*max = T[i];
				occurence = 1;
			}else{
				*max = T[j];
				occurence = 1;
			}

		}else if(*max == T[i] || *max == T[j]){
			if(i == j)
				occurence++;
			else if(T[i] == T[j])
				occurence +=2;
			else
				occurence++;
		}
		i++;
		j--;
	}
	return(occurence);
}
		\end{lstlisting}
		
% Section documentation
\section{Documentation}	
	Pour plus d'information sur le code vous pouvez consulter le \href{html/index.html}{site internet} contenant la documentation doxygen.
\end{document}
